
%%% Local Variables:
%%% mode: LaTeX
%%% TeX-master: t
%%% End:
\documentclass{../bsuir-std}

\begin{document}
%%% Local Variables: 
%%% coding: utf-8
%%% mode: latex
%%% TeX-engine: xetex
%%% End:

%%% Титульный лист
\begin{titlepage}
\begin{center}
Министерство образования Республики Беларусь\\[1.2em]
Учреждение образования\\[0.4em]
БЕЛОРУССКИЙ ГОСУДАРСТВЕННЫЙ УНИВЕРСИТЕТ ИНФОРМАТИКИ И РАДИОЭЛЕКТРОНИКИ\\[2.0em]
\end{center}

\vfill
\begin{center}
\textbf{ОТЧЕТ}\\
к лабораторной работе №3 на тему:\\
\textbf{ОПРЕДЕЛЕНИЕ НАДЕЖНОСТИ ТЕХНИЧЕСКОЙ СИСТЕМЫ}\\
\textbf{МЕТОДОМ ПРЯМОГО ПЕРЕБОРА ЕЁ РАБОТОСПОСОБНЫХ СОСТОЯНИЙ}\
\end{center}

\vfill
\begin{flushright}
    \begin{minipage}{9.3cm}
        Студент:  гр. 112601 Корякин~А.~Л.\\[0.1em]

        Отчет представлен на проверку: \underline{\hspace*{1.4cm}}.\underline{\hspace*{1.4cm}}.2024\\
        \underline{\hspace*{5.6cm}}
        
    \end{minipage}
  \end{flushright}
  
  \vfill
  \begin{center}
    {\normalsize Минск 2024}
    \end{center}

  \end{titlepage}

\graphicspath{ {img/} }

\section*{Цель лаборатроной работы}

Определение вероятности безотказной работы электронной системы
безопасности (ЭСБ) как разновидности технической системы методом
прямого перебора её работоспособных состояний.

\section{Результаты выполнения}

\begin{figure}[H]
  \centering
  \includegraphics[scale=0.5]{struct-and-table.png}
  \caption{Структурная схема исследуемой ЭСБ (скриншот)}
\end{figure}


ЭСБ будет работоспособной при работе ПКУ1, работе хотя бы одного из
устройств Д1 или Д2, работе хотя бы одного из устройств И1, И2, И3.


\begin{figure}[H]
  \centering
  \includegraphics[scale=0.5]{scheme.png}
  \caption{Схема расчёта надёжности (скриншот)}
\end{figure}

\begin{table}[H]
  \centering
\begin{tabular}[H]{| c | c | c | c | c | c |}
  \hline
   Д1 & Д2 & ПКУ1 & И1 & И2 & И3 \\ \hline
   1  & 1  & 1    & 1  & 1  & 1  \\ \hline
   0  & 1  & 1    & 1  & 1  & 1  \\ \hline
   1  & 0  & 1    & 1  & 1  & 1  \\ \hline
   1  & 1  & 1    & 0  & 1  & 1  \\ \hline                                 
   1  & 1  & 1    & 1  & 0  & 1  \\ \hline
   1  & 1  & 1    & 1  & 1  & 0  \\ \hline
   0  & 1  & 1    & 0  & 1  & 1  \\ \hline
   0  & 1  & 1    & 0  & 0  & 1  \\ \hline
   0  & 1  & 1    & 1  & 0  & 1  \\ \hline
   0  & 1  & 1    & 1  & 0  & 0  \\ \hline
   1  & 0  & 1    & 0  & 1  & 1  \\ \hline
   1  & 0  & 1    & 0  & 0  & 1  \\ \hline
   1  & 0  & 1    & 1  & 0  & 0  \\ \hline
  1  & 0  & 1    & 1  & 0  & 1  \\ \hline
\end{tabular}
\caption{Таблица состояний ЭСБ включенных в рабочее подмножество:
  подмножество работоспособных состояний ЭСБ}
\end{table}

Значение итогового показателя надёжности, полученного методом перебора
состояний ЭСБ и рассчитанного по СРН: $1 - 0,8674 = 0,133$

\section*{Выводы}
Методом прямого перебора работоспособных состояний была определена
вероятность безотказной работы электронной системы безопасности.

\end{document}
