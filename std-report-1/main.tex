%%% Local Variables: 
%%% compile-command: "sh build.sh" 
%%% End: 
%%% coding: utf-8
%%% mode: latex
%%% TeX-engine: xetex
%%% End:

% main.tex файл который компилируется и подключает за собой остальный
% файлы с содержимым через input
% При этом согласно моей же логике преамбула должна отстаться здесь.

% Используем опцию emptystyle т.к. bsuir-std расширяет eskdxtext,
% который без этой опции сделает рамку для чертежа даже на документе
% с просто текстом.
\documentclass[a4paper]{../bsuir-std}


\usepackage[sorting=none,backend=biber]{biblatex}
% библиогорафия в last
\usepackage{url}

\usepackage{float}

\begin{document}
% Путь к месту где картинки лежат
\graphicspath{ {..} }
% https://tex.stackexchange.com/a/139403

% Содержание
\tableofcontents
\newpage

%%% Введение.
\section*{Введение}
\addcontentsline{toc}{section}{Введение}

Настоящий отчет подготовлен по итогам производственной практики,
проходившей на базе
[название экологической организации или научного учреждения].
Целью практики было,
изучение влияния загрязнения окружающей среды на биоразнообразие,
выявление методов его сохранения.

\section{Цели и задачи практики}

Основные цели прохождения производственной практики включали:
\begin{enumerate}
  \item Оценка влияния загрязняющих факторов на экосистемы.
  \item Разработка ремомендаций по улучшению состояния окржающей среды.
\end{enumerate}

Задачи практики:
\begin{enumerate}
\item Сбор и анализ данных о состоянии экосистем.
\item Наблюдение за живыми организмами в различных условия  
\item Выполнение практических заданий таких как:
  \begin{enumerate}
  \item Проведение полевых исследований на объектах
  \item Подготовка отчетов о результатах исследований.
  \item Ознакомление с методами мониторинга экологической ситуации.
  \end{enumerate}
\end{enumerate}

\section{Описание предприятия}

Название экологической организации — это
краткое описание организации её сфера
деятельности история и т.д.
За время существования организация зарекомендовала себя как
[указать достижения или особенности в области охраны окружающей среды].

\section{Основные результаты практики}

Во время практики были достигнуты следующие результаты:
\begin{enumerate}
\item Овледение методами сбора и анализа экологичесеких данных;
\item Изучение воздействия различных факторов на биоразнообразие;
\end{enumerate}

Для анализа полученных данных использовалась следующая формула:

\begin{equation}
  G = {A \cdot B}{C}
\end{equation}

Где:
\begin{itemize}
\item A — первое число;
  
\item B — второе число;
  
\item C — общий знаменатель.
\end{itemize}

\section*{Заключение}

Производственная практика на [Название организации] оказала
положительное влияние на мои знания о состоянии окружающей среды и
подходах к её защите. Я получил ценный опыт, который поможет мне в
будущей карьере в области экологии охраны окружающей среды.



\section*{Приложения}
\addcontentsline{toc}{section}{Приложения}


[График биоразнообразия]

\begin{table}[H]
  \centering
  \begin{tabular}[c]{| l | c | r |}
    \hline
     Показатель & Значение 1 & Значние 2 \\ \hline
     Показатель A & X1 & Y1 \\ \hline
     Показатель B & X2 & Y2 \\ \hline
  \end{tabular}
  \caption{Таблица с результатами исследований.}
\end{table}


\end{document}
