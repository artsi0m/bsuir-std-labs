\documentclass[a4paper]{../bsuir-std}
\usepackage{listings}
%\usepackage{courier}
\lstset{basicstyle=\footnotesize\ttfamily,breaklines=true}

\usepackage[sorting=none,backend=biber]{biblatex}
\addbibresource{lab1.bib}


\begin{document}

%%% Local Variables: 
%%% coding: utf-8
%%% mode: latex
%%% TeX-engine: xetex
%%% End:

%%% Титульный лист
\begin{titlepage}
\begin{center}
Министерство образования Республики Беларусь\\[1.2em]
Учреждение образования\\[0.4em]
БЕЛОРУССКИЙ ГОСУДАРСТВЕННЫЙ УНИВЕРСИТЕТ ИНФОРМАТИКИ И РАДИОЭЛЕКТРОНИКИ\\[2.0em]
\end{center}

\vfill
\begin{center}
\textbf{ОТЧЕТ}\\
к лабораторной работе №3 на тему:\\
\textbf{ОПРЕДЕЛЕНИЕ НАДЕЖНОСТИ ТЕХНИЧЕСКОЙ СИСТЕМЫ}\\
\textbf{МЕТОДОМ ПРЯМОГО ПЕРЕБОРА ЕЁ РАБОТОСПОСОБНЫХ СОСТОЯНИЙ}\
\end{center}

\vfill
\begin{flushright}
    \begin{minipage}{9.3cm}
        Студент:  гр. 112601 Корякин~А.~Л.\\[0.1em]

        Отчет представлен на проверку: \underline{\hspace*{1.4cm}}.\underline{\hspace*{1.4cm}}.2024\\
        \underline{\hspace*{5.6cm}}
        
    \end{minipage}
  \end{flushright}
  
  \vfill
  \begin{center}
    {\normalsize Минск 2024}
    \end{center}

  \end{titlepage}


\graphicspath{ {img/docker-shots/} {img/portainer-shots/}}

\section{Пошаговое описание выполняемых действий}

Первым делом было создано персистентное хранилище.
В данном случае слово персистентный означает постоянство
данных в контейнере между запусками разных контейров в docker.

Затем была запущена комманда docker run взятая из документации
portainer community edition ~\cite{portainer}.

%\setmonofont{Iosevka}
\setmonofont{Courier New}
\begin{verbatim}
docker run -d \
  -p 8000:8000 -p 9443:9443 \
  --name portainer --restart=always \
  -v /var/run/docker.sock:/var/run/docker.sock \
  -v portainer_data:/data portainer/portainer-ce:2.21.1
\end{verbatim}

\begin{figure}[H]
  \centering
  \includegraphics[scale=0.2]{Screenshot-2024-09-17-20-54-58.png}
  \caption{Скриншот вышеприведенных комманд запущенных в терминале}
\end{figure}

Разберу пошагово запущенную команду.
Глагол run в аргументе командной строки у docker,
соответсвенно странице мануала означает
«создать и запустить контейнер из образа (image)».

Опция \texttt{-p} или её более длинный вариант \texttt{--publish}
означает сделать доступными на соотвествующих портах TCP/ip
хоста порты контейнера.

Опция \texttt{--name} позволяет указать имя для контейнера.

Опция \texttt{--restart=always} говорит docker
всегда перезапускать контейнер при выходе из строя.

Опция \texttt{-v} или \texttt{--volume} позволяет указать
персистнное хранилище, которое в дальшнейшем будет использовать контейнер.

Описание всех этих опция приведено в странице манула  \texttt{docker run}
~\cite{docker-run-manpage}.
После описанного шага бала осущствелена регистрация и
аутентификация в веб интерфейсе portainer.

\begin{figure}[H]
  \centering
  \includegraphics[scale=0.1]{Screenshot-2024-09-17-20-57-09.png}
  \caption{Страница входа в веб-интерфейс portainer в браузере}
\end{figure}

После этого я подключился к окружению local в portainer.

\begin{figure}[H]
  \centering
  \includegraphics[scale=0.15]{Screenshot- 2024-09-17-22-06-17.png}
  \caption{Результат запуска команды pull в portainer}
\end{figure}

Был скачан и запущен образ контейнера \texttt{cookiecutter},
скрипта для быстрого создания проектов с исходным кодом по шаблону.

\newpage

\section{Описание самого непонятного или сложного шага}

Самый непонятный для меня шаг был не в самостоятельной,
а в аудиторной части лабораторной работы, когда не удалось наладить
взаимодествие podman, скрипта обёртки podman-docker,
который позволяет работать с podman как docker и docker hub.

Почему-то не получалось сделать push на docker hub из podman.

Данную проблема решена не была,
вместо этого был удалён podman и настроен docker в соответсвии с
инструкцией используемого мною дистрибутива~\cite{suse-wiki-docker}.

\section{вывод}

В процессе выполнения лабораторной работы был развернут и запущен
образ Portainer.io Community Edition в режиме Docker Standalone.

В рамках работы был установлен контейнер с программным обеспечением
через интерфейс Portainer.io, а затем он был успешно удалён.

Также были проанализированы возможности управления контейнерами и
мониторинга их состояния, что позволило глубже понять функционал
Portainer.io и его применение в реальных сценариях.

\renewcommand{\refname}{\textbf{Cписок использованных источников}}
\DeclareFieldFormat{url}{Режим доступа\addcolon\space\url{#1} [Электронный ресурс].}
\DeclareFieldFormat{title}{{#1}}
\DeclareFieldFormat{labelnumberwidth}{[{#1}]\adddot  }

\addcontentsline{toc}{section}{Список использованных источников}
\printbibliography

\end{document}
