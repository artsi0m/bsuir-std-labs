%%% Local Variables:
%%% mode: LaTeX
%%% TeX-master: t
%%% End:

\documentclass[a4paper]{bsuir-tor}


%\usepackage[sorting=none,backend=biber]{biblatex}

\begin{document}

%%% Local Variables: 
%%% coding: utf-8
%%% mode: latex
%%% TeX-engine: xetex
%%% End:

%%% Титульный лист
\begin{titlepage}
\begin{center}
Министерство образования Республики Беларусь\\[1.2em]
Учреждение образования\\[0.4em]
БЕЛОРУССКИЙ ГОСУДАРСТВЕННЫЙ УНИВЕРСИТЕТ ИНФОРМАТИКИ И РАДИОЭЛЕКТРОНИКИ\\[2.0em]
\end{center}

\vfill
\begin{center}
\textbf{ОТЧЕТ}\\
к лабораторной работе №3 на тему:\\
\textbf{ОПРЕДЕЛЕНИЕ НАДЕЖНОСТИ ТЕХНИЧЕСКОЙ СИСТЕМЫ}\\
\textbf{МЕТОДОМ ПРЯМОГО ПЕРЕБОРА ЕЁ РАБОТОСПОСОБНЫХ СОСТОЯНИЙ}\
\end{center}

\vfill
\begin{flushright}
    \begin{minipage}{9.3cm}
        Студент:  гр. 112601 Корякин~А.~Л.\\[0.1em]

        Отчет представлен на проверку: \underline{\hspace*{1.4cm}}.\underline{\hspace*{1.4cm}}.2024\\
        \underline{\hspace*{5.6cm}}
        
    \end{minipage}
  \end{flushright}
  
  \vfill
  \begin{center}
    {\normalsize Минск 2024}
    \end{center}

  \end{titlepage}


\section{Общие сведения о разработке}

  \subsection{На что направлена разработка}

  Опытно-конструкторская разработка
  «Система тестирования сервоприводов квадрокоптера»
  направлена на создание системы,
  способной проводить тестирование работы одновременно
  четырёх сервоприводов и, опционально, полётного контроллера
  квадрокоптера.

  \subsection{Сведения о мировом уровне данного вида продукции.}
  
  В обиходе принято более краткое название этого устройства — сервотестер.
  По состоянии на сентябрь 2024 года, такие изделия изготавливаются
  промышленостью и доступны для приобритения.
  Основные функции этих устройств: тестирование сервоприводов,
  управление широтой импульсов широтно-импульсной модуляции,
  тестирование и настройка регуляторов оборотов электродвигателя и
  полетных контроллеров, а также измерение питающего напряжения.
  \subsection{Основные аспекты разработки: }
  \subsubsection{Технологические достижения: }
  c развитием микроконтроллерных устройств, многими компаниями были
  разработаны компактные и высокоточные сервотестеры.
  \subsubsection{Функциональные возможности: }
  Устройства могут обеспечивать как тестирование непосредственно
  сервоприводов, так и полетных контроллеров и регуляторов оборотов
  электродвигателя, подключенных к ним.
  \subsubsection{Применение: }
  Эти устройства находят примерние во всех тех областях, в которых уже
  так или иначе применяются устройства с сервоприводами, осущствляется и
  их сборка или ремонт. Например, в промышленных роботах и манипуляторах,
  автоматизированных станках, бытовых роботах, квадрокоптерах.
  \subsubsection{Стандарты и соответствие: }
  разработка подобных устройств также
  сопровождается необходимостью соблюдения различных международных
  стандартов, таких как ISO и IEC, что гарантирует их безопасность и
  надежность при эксплуатации.
  \subsubsection{Будущие тенденции: }
  ожидается, что с ростом устройств содержащих в себе сервоприводы,
  сервотестеры будут более востребованы. Рост спроса на этом рынке
  потребует внесения улучшений на уровне схемотехники изделия, для
  сохрания конкурентоспособности на рынке.
  \subsection{К компаниям,
  которые занимаются разработкой систем тестирования сервоприводов
  квадрокоптера можно отнести: }
  \subsubsection*{GSMIN} \  — компания производит сервотестеры для
  использования совместно со средой разработки arduino.
  \subsubsection*{PIMNARA} \   — компания производит сервотестеры для
  тестирования одновременно трех сервомоторов, с возможностью
  переключения частоты широко импульсной модуляции.
  \subsubsection*{G.T.Power} \  — компания производит сервеотестреы для
  тестирования одновременно четырех сервомоторов, с возомжностью
  вывода информации о прохождении тестирования на семисегментный дисплей.

  Эти компании предоставляют широкий спектр оборудовангия технология,
  предназнченный для проведения тестирования сервоприводов на
  различных этапах производства электронных устройств содержащих их.
  
\subsection{Ожидаемые результаты}
\subsection{Возможность дальнейшего развития}

\section{Наименование и область применения}
\subsection{Наименование:}

\subsection{Область применения}

\subsection{Предусматривается использование изделия для экспорта.}



\section{Основание для разработки}
\section{Цель и назначение разработки}
\section{Технические требования}


\end{document}
