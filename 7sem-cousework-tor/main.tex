%%% Local Variables:
%%% mode: LaTeX
%%% TeX-master: t
%%% End:

\documentclass[a4paper]{bsuir-tor}


%\usepackage[sorting=none,backend=biber]{biblatex}

\begin{document}

%%% Local Variables: 
%%% coding: utf-8
%%% mode: latex
%%% TeX-engine: xetex
%%% End:

%%% Титульный лист
\begin{titlepage}
\begin{center}
Министерство образования Республики Беларусь\\[1.2em]
Учреждение образования\\[0.4em]
БЕЛОРУССКИЙ ГОСУДАРСТВЕННЫЙ УНИВЕРСИТЕТ ИНФОРМАТИКИ И РАДИОЭЛЕКТРОНИКИ\\[2.0em]
\end{center}

\vfill
\begin{center}
\textbf{ОТЧЕТ}\\
к лабораторной работе №3 на тему:\\
\textbf{ОПРЕДЕЛЕНИЕ НАДЕЖНОСТИ ТЕХНИЧЕСКОЙ СИСТЕМЫ}\\
\textbf{МЕТОДОМ ПРЯМОГО ПЕРЕБОРА ЕЁ РАБОТОСПОСОБНЫХ СОСТОЯНИЙ}\
\end{center}

\vfill
\begin{flushright}
    \begin{minipage}{9.3cm}
        Студент:  гр. 112601 Корякин~А.~Л.\\[0.1em]

        Отчет представлен на проверку: \underline{\hspace*{1.4cm}}.\underline{\hspace*{1.4cm}}.2024\\
        \underline{\hspace*{5.6cm}}
        
    \end{minipage}
  \end{flushright}
  
  \vfill
  \begin{center}
    {\normalsize Минск 2024}
    \end{center}

  \end{titlepage}


\section{Общие сведения о разработке}

\subsection{На что направлена разработка \newline}

  Опытно-конструкторская разработка
  «Система тестирования сервоприводов квадрокоптера»
  направлена на создание системы,
  способной проводить тестирование работы одновременно
  четырёх сервоприводов и, опционально, полётного контроллера
  квадрокоптера.

  \subsection{Сведения о мировом уровне данного вида продукции. \newline}
  
  В обиходе принято более краткое название этого устройства — сервотестер.
  По состоянии на сентябрь 2024 года, такие изделия изготавливаются
  промышленностью и доступны для приобретения.
  Основные функции этих устройств: тестирование сервоприводов,
  управление широтой импульсов широтно-импульсной модуляции,
  тестирование и настройка регуляторов оборотов электродвигателя и
  полетных контроллеров, а также измерение питающего напряжения.
  \subsection{Основные аспекты разработки: }
  \subsubsection{Технологические достижения: }
  c развитием микроконтроллерных устройств, многими компаниями были
  разработаны компактные и высокоточные сервотестеры.
  \subsubsection{Функциональные возможности: }
  Устройства могут обеспечивать как тестирование непосредственно
  сервоприводов, так и полетных контроллеров и регуляторов оборотов
  электродвигателя, подключенных к ним.
  \subsubsection{Применение: }
  Эти устройства находят применение во всех тех областях, в которых уже
  так или иначе применяются устройства с сервоприводами, осуществляется и
  их сборка или ремонт. Например, в промышленных роботах и манипуляторах,
  автоматизированных станках, бытовых роботах, квадрокоптера.
  \subsubsection{Стандарты и соответствие: }
  разработка подобных устройств также
  сопровождается необходимостью соблюдения различных международных
  стандартов, таких как ISO и IEC, что гарантирует их безопасность и
  надежность при эксплуатации.
  \subsubsection{Будущие тенденции: }
  ожидается, что с ростом устройств содержащих в себе сервоприводы,
  сервотестеры будут более востребованы. Рост спроса на этом рынке
  потребует внесения улучшений на уровне схемотехники изделия, для
  сохранения конкурентоспособности на рынке.
  \subsection{К компаниям,
  которые занимаются разработкой систем тестирования сервоприводов
  квадрокоптера можно отнести: }
  \subsubsection*{GSMIN} \  — компания производит сервотестеры для
  использования совместно со средой разработки arduino.
  \subsubsection*{PIMNARA} \   — компания производит сервотестеры для
  тестирования одновременно трех сервомоторов, с возможностью
  переключения частоты широко импульсной модуляции.
  \subsubsection*{G.T.Power} \  — компания производит сервотестеры для
  тестирования одновременно четырех сервомоторов, с возможностью
  вывода информации о прохождении тестирования на семисегментный дисплей.

  Эти компании предоставляют широкий спектр оборудования технология,
  предназначенный для проведения тестирования сервоприводов на
  различных этапах производства электронных устройств содержащих их.
  
  \subsection{Ожидаемые результаты}
  
  \subsubsection{Повышение надёжности: }
  снижение количества вышедших из строя сервоприводов за счет
  тщательного контроля.

  \subsubsection{Экономия ресурсов: }
  оптимизация затрат на производства за счет выявления дефектов и
  снижение численности отбраковки.
  
  \subsubsection{Расширение функциональности: }
  возможность использования устройства для различных типов изделий
  использующих сервоприводы, что увеличивает его универсальность.

  Данная разработка может сыграть ключевую роль в улучшении качества и
  надёжности продукции завязанной на использование сервоприводов.

  \subsection{Возможность дальнейшего развития}
  Разрабатываемая система тестирования сервоприводов микроконтроллера.
  предполагает возможность дальнейшего развития, обеспечивая высокую
  надёжности и функциональности изделий использующих сервоприводы.

\section{Наименование и область применения}
\subsection{Наименование: \newline}
Система тестирования сервоприводов квадрокоптера (в дальнешем устройство).

\subsection{Область применения}


  \subsubsection{Производственные линии: }
  При тестировании и калибровке сервоприводов в промышленных роботах.
  Проверке и настройке сервоприводов в станках с численно-программным
  управлением. Диагностики и обслуживании сервоприводов в промышленных
  линиях.
  
  \subsubsection{Научные исследования: }
  Тестирование сервоприводов в научных приборах и установках.
  Проверка сервоприводов в роботах, используемых для обучения и
  исследований.
 
  \subsubsection{Сельское хозяйство: }
  Тестирование сервоприводов в системах управления поливом.
  Проверка сервоприводов в роботах для сбора урожая и других операций.
  
  \subsubsection{Электроника и бытовая техника: }
  Проверка сервоприводов в бытовых роботах.
  
  \subsubsection{Логистика: }
  Тестирование сервоприводов в системах автоматического складирования
  и извлечения товаров.
  Проверка сервоприводов в сортировочных конвейерах и роботах.


  \subsubsection{Аэрокосмическая и оборонная промышленность: }
  Тестирование сервоприводов в системах управления полетом.

  \subsubsection{Медицинское оборудование: }
  Проверка сервоприводов в медицинских сканерах и других
  диагностических устройствах.


  \subsubsection{Сервисные центры: }
  для диагностики и ремонта устройств с дефектными сервомоторами.
  

\subsection{Предусматривается использование изделия для экспорта.}

\section{Основание для разработки}

\subsection{Разработка системы тестирования сервоприводов обоснована}
несколькими
факторами связанными с современными требованиями к качеству и
недёжности изделий, завязанных на использование сервоприводов,
основными из которых являются:

  \subsubsection{Повышение надежности оборудования: }
  cервотестер позволяет проводить регулярные проверки и диагностику
  сервоприводов, что помогает выявлять и устранять потенциальные
  проблемы до того, как они приведут к отказам.
  
  \subsubsection{Снижение затрат на обслуживание: }
  Раннее выявление неисправностей позволяет избежать дорогостоящих
  ремонтов и замены оборудования.
  
  \subsubsection{Использование сложных изделий: }
  современные устройства достаточно сложны и потому используют
  одновременно не один, а несколько сервоприводов в своей конструкции.
  Поэтому требуется сервотестер с возможность одновременного
  тестирования сразу несколькоих подлкюченных сервоприводов.

  \subsubsection{Улучшение производительности: }
  Регулярное тестирование и калибровка сервоприводов обеспечивают их
  оптимальную работу, что повышает общую производительность системы.
  
  \subsubsection{Упрощение диагностики: }
  Автоматизированные тесты и диагностические функции сервотестера
  позволяют быстро и точно определять причины неисправностей.

  \subsubsection{Снижение времени простоя: }
  Быстрое выявление и устранение проблем сокращает время простоя
  оборудования, что особенно важно для производственных процессов.

  \subsubsection{Обучение и подготовка персонала: }
  Сервотестер может использоваться для обучения сотрудников, что
  повышает их квалификацию и способность к самостоятельной диагностике и
  ремонту.

  \subsubsection{Интеграция с другими системами: }
  Современные сервотестеры могут интегрироваться с системами
  управления и мониторинга, что позволяет централизованно управлять
  состоянием оборудования.
  
  \subsubsection{Снижение человеческого фактора: }
  Автоматизация процессов тестирования и диагностики снижает вероятность
  ошибок, связанных с человеческим фактором.

  \subsubsection{Увеличение срока службы оборудования: }
  Регулярное обслуживание и тестирование продлевают срок службы
  сервоприводов и других компонентов.

  \subsubsection{Оптимизация запасов: }
  Знание состояния сервоприводов позволяет более эффективно управлять
  запасами запчастей и расходных материалов.

  \subsubsection{Снижение эксплуатационных расходов: }
  Оптимальная работа сервоприводов снижает энергопотребление и другие
  эксплуатационные расходы.


  \subsubsection{Минимизация времени на тестирование: }
  устройство, свобоное проводить испытание сразу нескольких
  сервоприводов, позволяет сократить время на тестирование и отладку
  изделий, что важно в условиях быстрого выхода продукции на рынок.

  \subsubsection{Соответствие стандартам и нормативам: }
  Сервотестер помогает обеспечивать соответствие оборудования
  различным стандартам и нормативам, что особенно важно для
  регулируемых отраслей.

  \subsubsection{Улучшение клиентского сервиса: }
  Быстрая диагностика и ремонт оборудования повышают удовлетворенность
  клиентов и улучшают их взаимодействие с компанией.

  \section{Цель и назначение разработки}
  \subsection{Целью разработки является }
  создание системы тестирования сервоприводов квадрокоптера, позволяющую
  проводить диагностику, тестирование и калибровку сервоприводов для
  обеспечения их надежной и эффективной работы.

  \subsection{Основные задачи которые должны быть решены: }
  
  \subsubsection{Создать устройство для тестирования сервоприводов}

  \subsubsection{Проектирование схемы: }
  Разработка электрической схемы и печатной платы, обеспечивающих
  корректную работу всех компонентов устройства.

  \subsubsection{Подбор подходящих электронных компонентов, }
  микроконтроллерt, датчиков и других элементов, необходимых для
  реализации функций сервотестера.

  \subsubsection{Оптимизировать процессы}
  контроля и диагностики для повышения точности и эффективности.
  
  \subsubsection{Обеспечить возможность сбора и анализа}
  данных о работе сервотестера в реальном времени.

  \subsubsection{Улучшить надежность}
  и качество производимых изделий содержащих сервоприводы,
  через ранее выявлени проблем.

  В результате опытно-конструкторсой работы планируется
  создать инструмент, который обеспечит точный контроль
  работы серповриводов, что очень важно
  сразу для нескольких областей промышленности.

  \subsection{ Назначение разработки создание }
  конструктивно законченного
  устройства на базе современных изделий электронной техники.

  \subsection{ Разработка должна обеспечить }
  создание базовой модели устройства сервотестера.

  \subsection{ Дальнейшее развитие}
  разработки должно выполняться путем создания модификации базовой
  модели, отличающихся конфигурацией и изменениями функций на
  основе частных технических заданий.

  \subsection{Изделие предназначено для серийного изготовления.}

\section{Источники разработки}
\section{Технические требования}


\end{document}
