
%%% Local Variables:
%%% mode: LaTeX
%%% TeX-master: t
%%% End:
\documentclass{../bsuir-std}

\begin{document}

\graphicspath{ {img/} }
%%% Local Variables: 
%%% coding: utf-8
%%% mode: latex
%%% TeX-engine: xetex
%%% End:

%%% Титульный лист
\begin{titlepage}
\begin{center}
Министерство образования Республики Беларусь\\[1.2em]
Учреждение образования\\[0.4em]
БЕЛОРУССКИЙ ГОСУДАРСТВЕННЫЙ УНИВЕРСИТЕТ ИНФОРМАТИКИ И РАДИОЭЛЕКТРОНИКИ\\[2.0em]
\end{center}

\vfill
\begin{center}
\textbf{ОТЧЕТ}\\
к лабораторной работе №3 на тему:\\
\textbf{ОПРЕДЕЛЕНИЕ НАДЕЖНОСТИ ТЕХНИЧЕСКОЙ СИСТЕМЫ}\\
\textbf{МЕТОДОМ ПРЯМОГО ПЕРЕБОРА ЕЁ РАБОТОСПОСОБНЫХ СОСТОЯНИЙ}\
\end{center}

\vfill
\begin{flushright}
    \begin{minipage}{9.3cm}
        Студент:  гр. 112601 Корякин~А.~Л.\\[0.1em]

        Отчет представлен на проверку: \underline{\hspace*{1.4cm}}.\underline{\hspace*{1.4cm}}.2024\\
        \underline{\hspace*{5.6cm}}
        
    \end{minipage}
  \end{flushright}
  
  \vfill
  \begin{center}
    {\normalsize Минск 2024}
    \end{center}

  \end{titlepage}


\section{Пошаговое описание выполняемых действий}

\subsection*{Написание yaml файла для docker compose}

Был написан yaml файл следующего содержания:
\begin{verbatim}
version: "3"

services:
  karakin:
    image: "postgres:17-alpine"
    restart: always
    environment:
      - POSTGRES_PASSWORD=admin
      - POSTGRES_USER=admin
      - POSTGRES_DATABASE=admin
    volumes:
      - friday:/var/lib/postgresql/data
      - .:/csv
      
  artsi0m:
    image: "dpage/pgadmin4"
    ports:
      - 15432:15432
      - 80:5050
    environment:
      ALLOW_SPECIAL_EMAIL_DOMAINS: "true"
      PGADMIN_DEFAULT_EMAIL: admin@admin.com
      PGADMIN_DEFAULT_PASSWORD: artsi0m         
    volumes:
      - pgadmin_data:/var/lib/pgadmin
    depends_on:
      - karakin


volumes:
  friday:
  pgadmin_data:
\end{verbatim}

Разберём последовательно данный файл конфигурации:
\begin{enumerate}
\item В самом начале указана версия файла конфигурации
\verb|version: "3"|,
с которой будет работать docker compose.
Строка необходима в любом файле docker-compose.
Без неё воссоздать конфигурацию из файла будет невозможно;

\item После строки \texttt{services:} указаны два сервиса, которые
будут запущены \texttt{karakin} для контейнера содержащего СУБД и
\texttt{artsi0m} для контейнера содержащего веб-интерфейс для
управления СУБД;
  
\item Строкой \texttt{image:} указывается контейнер из Docker Hub, на
базе которого будет построен конкретный сервис;

\item \texttt{environment:} позволяет указать переменные окружения
  устанавливаемые для процессов внутри контейнера;

  
\item \texttt{volumes:} перечисляет персистентные, то есть существующие
  после перезапуска контейнера, хранилища подключаемые к контейнеру.
  Однако их предварительно нужно создавать командой \texttt{docker volume create}.
  Перечисление записывается через двоеточие, перед двоеточием каталог
  на хосте или подключаемое персистентное хранилище, а после двоеточия
  каталог в контейнере;

\item \verb|depends_on:| указывает зависимость одного сервиса от
  другого для последовательного запуска;

\item \texttt{ports:} позволяет указать какой порт будет соотвествовать
какому порту при трансляции портов осуществляемой docker для
соединений между хостом и контейнерами;

\item \texttt{restart: always} указывает всегда перезапускать
контейнер, если происходит какая-то ошибка не дающая контейнер
развернуть.

\end{enumerate}

\subsection*{Заполнение базы данных через веб-интерфейс pgadmin}

Для того, чтобы подключиться к \textit{pgadmin} были введены логин и
пароль перечисленные в переменных окружения в описании сервиса
\texttt{artsi0m}.

После этого было осуществлено подключение к СУБД из самого интерфейса
\textit{pgadmin}. 

\begin{figure}[H]
  \centering
  \includegraphics[scale=0.2]{pgadmin-connect-window.png}
  \caption{Окно подключение к СУБД в pgadmin}
\end{figure}


После этого в окне для запросов \textit{pgadmin} был
выполнен следующий запрос:
\begin{verbatim}
CREATE TABLE IF NOT EXISTS public.crypto
(
    slug character varying COLLATE pg_catalog."default",
    symbol character varying COLLATE pg_catalog."default",
    name character varying COLLATE pg_catalog."default",
    date date,
    ranknow integer,
    open numeric,
    high numeric,
    low numeric,
    close numeric,
    volume numeric,
    market numeric,
    close_ratio numeric,
    spread numeric
)
\end{verbatim}

Этот запрос создал таблицу \verb|crypto| и все соответствующие столбцы
в ней, необходимые для импорта значений из файла \verb|csv|.

После была выбрана опция импорта данных в веб-интерфейсе.
\begin{figure}[H]
  \centering
  \includegraphics[scale=0.1]{pgadmin-import-export.png}
  \caption{Опция импорта данных}
\end{figure}

И был загружен файл \verb|csv|
\begin{figure}[H]
  \centering
  \includegraphics[scale=0.2]{pgadmin-csv-import.png}
  \caption{Окно импорта данных}
  
\end{figure}

Была также выбрана опция \verb|Header| на вкладе \verb|Options|
потому что в импортируемом файле первая строка начинается с
подписей для столбцов.

\section{Описание самого непонятного или сложного шага}

Самым непонятным было то, что почему-то сервисы СУБД и веб интерфейса для
управления СУБД были размещены не на локальном адресе 127.0.0.1,
а в подсети для частного использования 172.20.0.1/24.

\begin{figure}[H]
  \centering
  \includegraphics[scale=0.1]{portainer-subnet.png}
  \caption{Скриншот portainer демонстрирующий адрес запущенного контейнера }
  
\end{figure}

Кроме этого каких-либо препятствий для выполнения работы не возникло.

\section{Вывод}

Были изучены методы разворачивания сервисов при помощи
\textit{docker compose} и использован функционал персистентных хранилищ
\textit{docker volume}.

\end{document}
